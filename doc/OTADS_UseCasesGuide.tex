% 
% Permission is granted to copy, distribute and/or modify this document
% under the terms of the GNU Free Documentation License, Version 1.2
% or any later version published by the Free Software Foundation;
% with no Invariant Sections, no Front-Cover Texts, and no Back-Cover
% Texts.  A copy of the license is included in the section entitled "GNU
% Free Documentation License".




%%%%%%%%%%%%%%%%%%%%%%%%%%%%%%%%%%%%%%%%%%%%%%%%%%%%%%%%%%%%%%%%%%%%%%%%%%%%%%%%%%%%%%%%%% 
\section{Use Cases Guide}

This section presents the main features of the module $otads$ in their context.\par

%%%%%%%%%%%%%%%%%%%%%%%%%%%%%%%%%%%%%%%%%%%%%%%%%%%%%%%%%%%%%%
\subsection{The basic adaptive directional sampling algorithm: ADS-2}

The following Python script shows how the basic adaptive directional sampling algorithm can be used.\par

\begin{lstlisting}
import openturns as ot
from otads import AdaptiveDirectionalSampling

# Event definition: R - S (see OpenTURNS main documentation)
model = ot.NumericalMathFunction(ot.Description(['R', 'S']),
                                 ot.Description(['G']),
                                 ot.Description(['R - S']))
input = ot.ComposedDistribution(
  ot.DistributionCollection([ot.Normal(5., 1.),
                             ot.Normal(3., 1.)]))
output = ot.RandomVector(model,
                         ot.RandomVector(input))
event = ot.Event(output, ot.LessOrEqual(), 0.)
                                 
# Create ADS-2 algorithm
# (other constructors exist, see AdaptiveDirectionalSampling.__doc__)
algorithm = AdaptiveDirectionalSampling(event)

# Set the total computational budget (total number of directions)
algorithm.setMaximumOuterSampling(int(1e4))

# Set fractions of the total computational budget for the learning and
# estimation steps (should sum up to 1)
algorithm.setGamma([.5, .5])

# Set other directional sampling options (see OpenTURNS main documentation)
algorithm.setBlockSize(int(1e2))
algorithm.setMaximumCoefficientOfVariation(.2)
algorithm.setRootStrategy(ot.SafeAndSlow())
algorithm.setSamplingStrategy(ot.RandomDirection(2))

# Run & get results
algorithm.run()
results = algorithm.getResult()
print(results)
\end{lstlisting}

%%%%%%%%%%%%%%%%%%%%%%%%%%%%%%%%%%%%%%%%%%%%%%%%%%%%%%%%%%%%%%
\subsection{The dimension reduction variant of the adaptive directional sampling algorithm: ADS-2+}

The following Python script shows how to enable the dimension reduction variant.\par

\begin{lstlisting}
import openturns as ot
from otads import AdaptiveDirectionalSampling

# Event definition: R - S with dummy additional variables
# (see OpenTURNS main documentation)
model = ot.NumericalMathFunction(ot.Description(['R', 'S', 'dummy1', 'dummy2']),
                                 ot.Description(['G']),
                                 ot.Description(['R - S']))
input = ot.ComposedDistribution(
  ot.DistributionCollection([ot.Normal(5., 1.),
                             ot.Normal(3., 1.),
                             ot.Normal(),
                             ot.Normal()]))
output = ot.RandomVector(model,
                         ot.RandomVector(input))
event = ot.Event(output, ot.LessOrEqual(), 0.)

# Create ADS-2 algorithm
# (other constructors exist, see AdaptiveDirectionalSampling.__doc__)
algorithm = AdaptiveDirectionalSampling(event)
algorithm.setMaximumOuterSampling(int(1e4))

# Enable partial stratification (dimension reduction)
algorithm.setPartialStratification(True)

# Set the maximum number of stratified dimensions
algorithm.setMaximumStratificationDimension(2)

# Run & get results
algorithm.run()
results = algorithm.getResult()
print(results)
print('T = ', algorithm.getTStatistic())
\end{lstlisting}

%%%%%%%%%%%%%%%%%%%%%%%%%%%%%%%%%%%%%%%%%%%%%%%%%%%%%%%%%%%%%%
\subsection{The design point variant of the adaptive directional sampling algorithm: DP-ADS-2}

Assuming some OpenTURNS \emph{Event} has been defined, the following Python script shows how to enable the dimension reduction variant.\par

Python script for this use case:

\begin{lstlisting}
from otads import AdaptiveDirectionalSampling

# Event definition: R - S (see OpenTURNS main documentation)
model = ot.NumericalMathFunction(ot.Description(['R', 'S']),
                                 ot.Description(['G']),
                                 ot.Description(['R - S']))
input = ot.ComposedDistribution(
  ot.DistributionCollection([ot.Normal(5., 1.),
                             ot.Normal(3., 1.)]))
output = ot.RandomVector(model,
                         ot.RandomVector(input))
event = ot.Event(output, ot.LessOrEqual(), 0.)

# Preliminary FORM analysis (see OpenTURNS main documentation)
form = ot.FORM(ot.AbdoRackwitz(),
               event,
               input.getMean())
form.run()
form_results = form.getResult()

# Create ADS-2 algorithm
# (other constructors exist, see AdaptiveDirectionalSampling.__doc__)
algorithm = AdaptiveDirectionalSampling(event)
algorithm.setMaximumOuterSampling(int(1e4))

# Set the design point for rotation
algorithm.setQuadrantOrientation(form_results.getStandardSpaceDesignPoint())

# Run & get results
algorithm.run()
results = algorithm.getResult()
print(results)
\end{lstlisting}
