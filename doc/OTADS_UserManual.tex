% 
% Permission is granted to copy, distribute and/or modify this document
% under the terms of the GNU Free Documentation License, Version 1.2
% or any later version published by the Free Software Foundation;
% with no Invariant Sections, no Front-Cover Texts, and no Back-Cover
% Texts.  A copy of the license is included in the section entitled "GNU
% Free Documentation License".

%%%%%%%%%%%%%%%%%%%%%%%%%%%%%%%%%%%%%%%%%%%%%%%%%%%%%%%%%%%%%%%%%%%%%%%%%%%%%%%%%%%%%%%%%% 
\section{User Manual}

This section gives an exhaustive presentation of the objects and functions provided by the $otads$ module, in the alphabetic order.


\subsection{AdaptiveDirectionalSampling}

The AdaptiveDirectionalSampling inherits from the simulation class. Refer to the appropriate documentation for the methods of the Simulation class.

\begin{description}

\item[Usage :] \rule{0pt}{1em}
  \begin{description}
  \item $AdaptiveDirectionalSampling(event)$
  \item $AdaptiveDirectionalSampling(event, rootStrategy)$
  \item $AdaptiveDirectionalSampling(event, rootStrategy, samplingStrategy)$
  \end{description}

\item[Arguments :]  \rule{0pt}{1em}
  \begin{description}
  \item $event$ : an Event, the event we want to evaluate the probability
  \item $rootStrategy$ : a RootStrategy, the strategy adopted to evaluate the intersections of each direction with the limit state function and take into account the contribution of the direction to the event probability
  \item $samplingStrategy$ : a SamplingStrategy, the strategy adopted to sample directions.
  \end{description}

\item[Value :] an AdaptiveDirectionalSampling

\item[Details :]  \rule{0pt}{1em}
  \begin{description}
  \item AdaptiveDirectionalSampling constructor
  \end{description}

\item $getGamma$
    \begin{description}
    \item[Usage :] $getGamma()$
    \item[Arguments :] no argument
    \item[Value :] a vector, the repartition of the budget for each learning/evaluation step.
    \end{description}
    \bigskip

\item $getRootStrategy$
    \begin{description}
    \item[Usage :] $getRootStrategy()$
    \item[Arguments :] no argument
    \item[Value :] a RootStrategy, the strategy adopted to evaluate the intersections of each direction with the limit state function and take into account the contribution of the direction to the event probability.
    \end{description}
    \bigskip

\item $getSamplingStrategy$
    \begin{description}
    \item[Usage :] $getSamplingStrategy()$
    \item[Arguments :] no argument
    \item[Value :] a SamplingStrategy, the strategy adopted to sample directions.
    \end{description}
    \bigskip
    
\item $getQuadrantOrientation$
    \begin{description}
    \item[Usage :] $getQuadrantOrientation()$
    \item[Arguments :] no argument
    \item[Value :] a vector, the bisector of the first quadrant used for the DPADS variant.
    \end{description}
    \bigskip    
    
 \item $getPartialStratification$
    \begin{description}
    \item[Usage :] $getPartialStratification()$
    \item[Arguments :] no argument
    \item[Value :] a boolean deciding if the ADS-2+ variant is used, default [false].
    \end{description}
    \bigskip     
    
\item $getMaximumStratificationDimension$
    \begin{description}
    \item[Usage :] $getMaximumStratificationDimension()$
    \item[Arguments :] no argument
    \item[Value :] an integer, the maximum number of variables to be stratified in the ADS+ variant, default [3].
    \end{description}
    \bigskip 
    
  \item $getTStatistic$
    \begin{description}
    \item[Usage :] $getTStatistic()$
    \item[Arguments :] no argument
    \item[Value :] a vector, the values of the $\tilde T$ statistic.
    \end{description}
    \bigskip   
    
The methods $getXXXX$ methods have a corresponding $setXXX$ method, except for $getTStatistic$.

\item[Links] \rule{0pt}{1em}
\end{description}

